% !Mode:: "TeX:UTF-8"

\chapter{绪论}

\section{研究工作的背景与意义}
本论文依托于国家“973”项目“卫星故障演变规律与故障反演技术”。本项目基于两颗国防卫星,从2011年至2013年近两年半的数十种遥测信号,在时域、频域、时频联合域及其他变换域,对卫星在轨异变状态进行研究。项目主要研究内容包括,对卫星遥测信号异变状态的特征提取,分析不同特征参数的演变趋势,建立特征参数模型,最后对卫星状态异变进行短期预测。本项目旨在,通过对卫星遥测信号的分析,能够掌握卫星的健康状况,即卫星的异变状态,从而为卫星管理人员提供正确的决策参考。

\subsection{研究工作的背景}
本论文的研究工作,是在国内卫星系统不断扩大,对卫星运行的稳定性和卫星寿命水平的要求不断提高的同时提出的。
研究工作不仅仅要求对卫星故障进行诊断和预测,同时对卫星的异变状态的检测和预测也提出了要求。

卫星“状态异变”的含义较“故障信号”更为广泛。在卫星系统中,故障信号指的是卫星的机械装置或者电路系统已经不能完成预设的功能时产生的信号,即已经出现了功能性缺失\citeup{belov2005space};而状态异变包括故障信号,它还包括,在发生功能性缺失的前后,设备产生的不同于正常状态下的信号。在此要说明的是,异变信号的出现可能伴随着故障的生成,也有可能不会引起故障,即设备出现扰动,但是并未造成功能性缺失。
所以,相较于故障信号,状态异变更具有隐蔽性,随机性。这样对信号特征的提取,建模,乃至预测的难度都大大提高。在本论文及项目中,研究对象状态异变信号使用的都是实际的卫星遥测信号,由于遥测信号的接收、干扰等因素,信号的野值较多,质量较差。并且卫星遥测信号,大多具有非平稳的特性。这些因素对状态异变信号的分析和预测工作提出了更大的挑战。

\subsection{研究工作的意义}
%首先
首先,卫星造价昂贵,工艺复杂,制造周期长,且具有唯一性,一旦卫星因异变而损坏废弃,造成的损失将非常严重。所以研究卫星状态的异变并预测,对卫星行业具有极其重要的意义。

对于投资巨大和庞大复杂的卫星系统,可靠性和安全性是迫切需要的。否则,一次故障可能导致灾难性的的后果和巨大的经济损失。
如1986年1月美国“挑战者”号航天飞机失事导致7名宇航员全部遇难;1999年4月9日到5月4日的不到一个月的时间里,美国大力神4B、雅典娜2、德尔它2运载火箭等发射连续失败,损失达十亿美元;1999年11月日本H2运载火箭发射失败,损失2.99亿美元;2000年11月,印度INSA-2B卫星丧失对地指向功能,造成重大损失;2003年2月1日美国哥伦比亚号航天飞机在返回途中解体,机组人员全部遇难。
可见,对卫星系统的故障乃至状态异变进行诊断和预测,对航天事业的健康发展有着重要意义。

%其次
其次,目前我国的卫星在数量、型号、功能等各方面不断增加,加上国外势力对我国卫星的干扰等其他因素,传统的异变分析及预测方法将不再完全适用于目前的卫星状况。所以研究更可靠的预测算法必然是目前的趋势。

随着在轨卫星数量、设计寿命和型号种类的不断增加,维持其安全稳定运行变得越来越重要,在轨管理的难度也逐年加大。在轨卫星长期运行在空问环境中,受到多种不确定性因素的作用,其性能与功能可能会出现变化,反映在遥测参数上也会有些变化,如果在轨卫星发生异常,相应的遥测参数的变化趋势也会发生改变。因此,分析在轨卫星的遥测数据变化规律,选择相适应的数据预测方法,对遥测数据进行预测,并在此基础上实现预警,可以在早期及时发现遥测数据的异常变化,有效避免可能发生的重大故障,降低卫星在轨运行的风险,同时为异常的处理赢得宝贵时间,这对于提高卫星在轨运行的安全性和可靠性具有重要的意义\citeup{qin2010yizhong}。

%最后
最后,当前卫星异变状态分析和预测工作通常是一项耗时、重复和劳动强度很大的工作\citeup{yang2004jiyu}。所以研究自动化的卫星状态异变的预测算法将为卫星管理工作带来极大的便利。

在大部分情况下,卫星管理人员采用手工检查卫星遥测数据的方式来确定卫星是处于健康状态还是异变状态,以及卫星是否存在危险的趋势,以预测卫星是否在不久的将来处于故障状态。应用的具体方法或者是一些统计学估计方法,或者是通过比较卫星遥测参数的实测值与期望值之间的差异方法。这些方法存在严重的缺陷,主要是过分依赖人工经验以及难以实现故障诊断与预测的自动推理\citeup{yang2004jiyu}。
当卫星状态异变出现了新的模式,传统方法的预测结果就会出现较大偏差。研究自动化的卫星状态异变的预测方法,既为卫星管理人员提供了便利,更重要的是,在预测精度上有显著提升,减小预测的虚警和漏警。 

\section{国内外研究历史与现状}


\section{本文的主要贡献与创新}


\section{本文的结构安排}


\section{本章小结}