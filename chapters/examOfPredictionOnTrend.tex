% !Mode:: "TeX:UTF-8"

\chapter{基于趋势预测算法的实验仿真}
在第\ref{chap:topot}章中讲解了基于趋势预测算法的理论,而在本章中将根据第\ref{chap:topot}章中的理论,进行仿真实验的描述。

\section{实验方案及预测结果评判指标}
\subsection{实验方案}
在此先给出基于趋势预测算法的实验方案图,如图\ref{plan_pot}所示。
\pic[htbp]{基于趋势预测算法的实验方案}{}{plan_pot} 

基于趋势预测算法的实验方案,是采取算法融合的方式来对卫星在轨异变数据进行预测,主要用到小波分析算法,ARMA预测模型,SVR预测模型等方法。
根据实验方案图,下面将给出具体的算法步骤:
\begin{enumerate}[label=(\arabic*)]
	\item 根据实测卫星信号$Y:\{ y_1, y_2,\dots \}$的时频特征,选取一个小波对信号序列进行多尺度分解,对信号的趋势进行分析,确定分解尺度$n$和最大分解层数$J$。
	\item 信号经小波分解方法处理,得到高频序列$d_{n,i}$和低频序列$c_{J,i}$;
	\item 分别对高频序列$d_{n,i}$和低频序列$c_{J,i}$进行数据平稳性检验。一般情况下,高频序列是满足平稳性的,而低频序列则要视情况而定。
	\item 如果低频序列$c_{J,i}$和$d_{n,i}$满足平稳性验证,则对其经行ARMA预测,如果不满足,则对其进行SVR预测。
	\item 对得到的高频预测值$d_{n,i+1}^\prime$和低频预测值$c_{J,i+1}^\prime$进行小波重构,即进行加权计算,得到最终结果$y_{i+1}^\prime$:
	$$ y_{i+1}^\prime = d_{1,i+1}^\prime + d_{2,i+1}^\prime + \dots +d_{J,i+1}^\prime + c_{J,i+1}^\prime$$
	同理,根据$t_i$时刻之前的数据预测第$k$步状态值,则有:
	\begin{equation}
	\label{equ:yipk}
		y_{i+k}^\prime = d_{1,i+k}^\prime + d_{2,i+k}^\prime + \dots +d_{J,i+k}^\prime + c_{J,i+k}^\prime
	\end{equation}
\end{enumerate}

\subsection{预测结果评判指标}
一般时间序列预测结果的评判指标有以下三种方式:
\begin{enumerate}[label=(\arabic*)]
	\item 均方根误差:
	\begin{equation}
		\mathrm{RMSE} = \sqrt{{1\over n} \sum_{t=1}^{n}{(y_t-\hat{y_t})}^2}
	\end{equation}
	\item 平均绝对百分比误差:
	\begin{equation}
		\mathrm{MAPE} = {1\over n} \sum_{t=1}^{n}{\left| y_t-\hat{y_t} \right|  \over \left| y_t \right|  }
	\end{equation}
	\item 平均绝对误差:
	\begin{equation}
		\mathrm{MAE} = {1\over n} \sum_{t=1}^{n}{\left| y_t-\hat{y_t} \right|  }
	\end{equation}
\end{enumerate}

\section{实验流程}
\subsection{实验数据的选取}
本实验选取的数据是某卫星在2011年5月初某一天多内,转向轮子合成的Y轴角动量(Hy)的遥测数据。其时域图如图\ref{POT_shiyutu}所示。
\pic[htbp]{实验数据Hy信号时域图}{}{POT_shiyutu}

从图中可以看出,数据变化趋势不固定,并且数据毛刺较多。

本实验的目标是对这段数据进行在轨实时短期预测。

\subsection{小波分解}
接下来对数据进行小波分解,综合多种因素,本实验选取db5小波,阶数选择为7。下图是得到的小波分解结果如图\ref{POT_xiaobo}。
\pic[htbp]{Hy信号的小波分解图}{}{POT_xiaobo}

从图中可以看出,随着分解层数的提高,小波分解的低频分量越平滑,高频分量展现了一些规律性。

\subsection{对低频分量的平稳性检验及预测}
在本实验中需要对低频分量进行平稳性检验,如果信号满足平稳性,则可以使用ARMA预测算法,进行预测;如果信号不满足平稳性,则需要使用SVR预测算法,进行预测。在此采用Augmented Dickey-Fuller test(ADF)单位根验证方法,利用Eviews软件来实现。

首先给出卫星遥测信号Hy的小波分解低频分量HyA7的时域图如图\ref{POT_A7_shiyu}。
\pic[htbp]{Hy信号小波分解低频分量HyA7时域图}{}{POT_A7_shiyu}
从时域图可以看出,低频分量HyA7是非平稳的。

本实验对序列进行ADF检验。序列不存在明显的趋势,所以选择对常数项,不带趋势的模型进行检验,检验结果如表\ref{tab:HyA7_ADF} 所示。
\begin{table}[htbp] 
	\centering
	\caption{\label{tab:HyA7_ADF}HyA7的ADF检验结果表} 
	\begin{tabular}{ccc} 
		\toprule 
		{} & T统计量值 & P值\\ 
		\midrule
		ADF & -1.049255 &  0.7367 \\
		1\%显著性水平检验临界值 & -3.430904 & {}\\ 
		5\%显著性水平检验临界值& -2.861669 & {}\\
		10\%显著性水平检验临界值& -2.566880 & {}\\
		\bottomrule 
	\end{tabular} 
\end{table}
结果表明不能拒绝存在一个单位根的原假设,ADF统计值大于1\%level、5\%level和10\%level的临界值,序列是非平稳的。因此,本试验将采用非稳系统分析的SVR模型对低频分量进行建模预测。

以下是对低频分量HyA7进行SVR建模预测过程。全数据段为9000个数据点,取前500个数据点对未来的10个数据点进行预测,并且采用的是在轨的动态预测方法。在轨SVR预测结果图如图\ref{POT_A7_yuce} 所示。
\pic[htbp]{Hy信号小波分解低频分量HyA7时域图}{}{POT_A7_yuce}

在图中,蓝色的实线是原始的时域数据曲线,绿色的虚线是预测结果。可以从图中直观的看到预测结果比较理想。

\subsection{对高频分量的平稳性检验及预测}
在此,本实验取卫星遥测数据Hy的小波分解后的其中一个高频分量HyD1经行演示。HyD1的时域波形如图\ref{POT_D1_shiyu}所示。
\pic[htbp]{Hy信号小波分解低频分量HyA7时域图}{}{POT_D1_shiyu}

从图中可以看到,信号是满足平稳性的。并且卫星遥测信号的小波分解高频分量一般都具有平稳性的,但是在此,我们同样要使用Eviews进行平稳性检验,并来确定ARMA的模型参数。

试验中对高频分量HyD1进行ADF检验,检验结果如下表\ref{tab:HyD1_ADF}所示,ADF统计值远小于三个比较值,序列是平稳的,可以对其进行ARMA建模预测。
\begin{table}[htbp] 
	\centering
	\caption{\label{tab:HyD1_ADF}HyA7的ADF检验结果表} 
	\begin{tabular}{ccc} 
		\toprule 
		{} & T统计量值 & P值\\ 
		\midrule
		ADF & -50.83602 &  0.0001 \\
		1\%显著性水平检验临界值 & -3.430904 & {}\\ 
		5\%显著性水平检验临界值& -2.861669 & {}\\
		10\%显著性水平检验临界值& -2.566880 & {}\\
		\bottomrule 
	\end{tabular} 
\end{table}

接下来确定HyD1的ARMA模型参数。先求得HyD1高频分量的相关图,如图 \ref{hyd1xiangguan}所示。
\pic[htbp]{HyD1相关结果图}{}{hyd1xiangguan}

根据序列相关图中的自相关系数和偏相关系数估计模型参数,序列的自相关系数在3阶后截尾,ARMA模型参数$p$可以考虑选择2,偏自相关系数在5阶后拖尾,ARMA模型参数$q$可以考虑选择5。

但是在选择ARMA模型参数时,通过自相关系数和偏相关系数估计出的参数只能作为参考,而目前常用的信息准则为AIC、BIC、FPE等。AIC准则是最小信息准则,它的目的是判断预测目标的发展过程与哪一随机过程最为接近。在应用中,设置模型的阶数在一定范围内从低到高,对不同的阶数的模型计算AIC值,AIC值最小对应的阶数就是合适的阶数。

根据自相关系数和偏相关系数估计出的参数作为范围,算出AIC指标如表\ref{tab:ARMA_AIC}所示。
\begin{table}[htbp] 
	\centering
	\caption{\label{tab:ARMA_AIC}ARMA模型参数AIC结果表} 
	\begin{tabular}{ccc} 
		\toprule 
		$p$ & $q$ & AIC\\ 
		\midrule
		2 & 5 &  -9.572221 \\
		2 & 4 &  -9.405544 \\
		2 & 3 &  -9.402324 \\
		\textbf{2} & \textbf{2} &  \textbf{-9.672930} \\
		\textbf{2} & \textbf{1} &  \textbf{-9.971523} \\
		1 & 5 &  -9.437207 \\
		1 & 4 &  -9.546885 \\
		\textbf{1} & \textbf{3} &  \textbf{-9.626855} \\
		1 & 2 &  -9.511113 \\
		1 & 1 &  -9.299825 \\
		\bottomrule 
	\end{tabular} 
\end{table}

根据不同模型参数算出的AIC指标,得知模型ARMA(2,2)、ARMA(2,1)和ARMA(1,3)的AIC指标都比较小。经过进一步筛选和比较AIC指标,选取ARMA(2,1)模型,其估计结果如图\ref{HyD1ARMA21}所示。
\pic[htbp]{HyD1的ARMA(2,1)结果图}{}{HyD1ARMA21}

实验结果可以看出,系数高度显著。可以看出ARMA(2,1)模型是较优选择。确定模型后,对HyD1进行ARMA动态预测。预测结果如图\ref{POT_D1_yucebijiao}所示。
\pic[t]{高频分量HyD1及其预测值比较图}{}{POT_D1_yucebijiao}

根据以上的方法与流程,对其他高频分量HyD2至HyD7建立预测模型并进行短期预测。

得到的预测模型参数如表\ref{tab:gaopin_ARMA}所示。
\begin{table}[htbp] 
	\centering
	\caption{\label{tab:gaopin_ARMA}高频分量的预测模型参数表} 
	\begin{tabular}{cc} 
		\toprule 
		高频分量 & 预测模型参数 \\ 
		\midrule
		HyD2 &  ARMA(2,3) \\
		HyD3 &  ARMA(2,4) \\
		HyD4 &  ARMA(3,4) \\
		HyD5 &  ARMA(2,3) \\
		HyD6 &  ARMA(2,4) \\
		HyD7 &  ARMA(4,4) \\
		\bottomrule 
	\end{tabular} 
\end{table}

根据以上模型参数建立的ARMA模型对各个高频分量进行预测。得到的预测结果与原始时域值比较图如图\ref{pic:arma_comp}所示。
\begin{pics}[htbp]{Hy信号高频分量的预测值与原始值比较图}{pic:arma_comp}
	\addsubpic{HyD2及其预测值}{}{POT_D2_yucebijiao}
	\addsubpic{HyD3及其预测值}{}{POT_D3_yucebijiao}
	\addsubpic{HyD4及其预测值}{}{POT_D4_yucebijiao}
	\addsubpic{HyD5及其预测值}{}{POT_D5_yucebijiao}
	\addsubpic{HyD6及其预测值}{}{POT_D6_yucebijiao}
	\addsubpic{HyD7及其预测值}{}{POT_D7_yucebijiao}
\end{pics}
\clearpage

\subsection{对预测值进行小波重构}
根据式(\ref{equ:yipk})对得到的高频预测值$d_{n,i+1}^\prime$和低频预测值$c_{J,i+1}^\prime$进行小波重构,即进行加权计算,得到最终结果如图\ref{POT_Hy_yuce}所示。
\pic[htbp]{遥测信号Hy及其预测值Hyp}{}{POT_Hy_yuce}

计算得出该预测结果的均方根误差$\mathrm{RMSE} =0.0109$,平均绝对百分比误差:$\mathrm{MAPE}  =0.0236$,平均绝对误差$\mathrm{MAE} =0.0079$。

\section{实验结果比较、分析}
下面就一些其他方法和本论文方法结果进行比较。现行比较常用的方法有,将原始时域数据直接用SVR预测方法进行预测,或者将原始数据进行小波去噪之后,使用SVR方法进行预测。用这两者方法得到的预测结果图分别如图\ref{SVR_Hy_yuce}和图\ref{QZSVR_Hy_yuce}所示。
\pic[t]{遥测信号Hy及通过SVR方法预测值Hyp}{}{SVR_Hy_yuce}
\pic[htbp]{遥测信号Hy及通过去噪SVR方法预测值Hyp}{}{QZSVR_Hy_yuce}

比较上述两个方法和本实验方法的预测指标如表\ref{tab:fangfa_comp}所示。
\begin{table}[htbp] 
	\centering
	\caption{\label{tab:fangfa_comp}高频分量的预测模型参数表} 
	\begin{tabular}{lccc} 
		\toprule 
		预测方法 & RMSE & MAPE & MAE \\ 
		\midrule
		SVR预测方法 & 0.0160 & 0. 0347 & 0. 0115 \\ 
		去噪SVR预测方法 & 0.0132 & 0.0282 & 0.0101 \\ 
		本论文方法:基于趋势预测方法 & 0.0109 & 0.0236 & 0.0079 \\ 
		\bottomrule 
	\end{tabular} 
\end{table}

从预测指标看来,本论文的方法在每项指标上都比其他两个预测模型,预测的结果要好不少。
\section{本章小结}
本章主要对基于趋势的预测算法的仿真实验进行描述。

首先,提出了基于趋势的预测算法的实验方案预测结果的评判指标;
接下来,逐步讲解了仿真实验的每一步操作,并给出了逐步实验结果;
最后给出了本论文算法方案与其他预测方法结果的比较,并进行分析。

通过对实际卫星遥测数据的仿真实验,可以看出基于趋势的预测算法具有优良的现实效果,对理论算法是一次有效的验证。